\documentclass[12pt,letterpaper]{article}

\usepackage[left=1in,right=1.0in,top=1.0in,bottom=1.0in,
           includehead=true,headsep=1.0in,includefoot=true]{geometry}
\usepackage{setspace}
\usepackage{url}
\usepackage{tikz}
\usetikzlibrary{positioning,shapes,arrows}

\title{
  Clinical Decision Support for OpenMRS
}
\author{
        Bierman, Robert,  \emph{Group Lead}  \\ \texttt{bierman@mail.sfsu.edu} \and 
        Woeltjen, Victor, \emph{Group Lead}  \\ \texttt{woeltjan@mail.sfsu.edu} \and
        Choi, Kay       \and
        Gimeno, Steven  \and
        Lum, Jason      \and
        Ng, Ying Kit    \and
        Uy, Bianca      
} 



\begin{document}


\newpage 

\maketitle
\begin{center}
\begin{Large}\emph{Group 1:} Final Project for CSC 668-868 Spring 2013\end{Large} \linebreak
\url{https://code.google.com/p/sp2013-csc668-868-group1/}
\end{center}
\thispagestyle{empty} % Suppress page number

\newpage \pagenumbering{roman}
\tableofcontents


\newpage \pagenumbering{arabic}
\section{Contributions} 

\subsection{Contributions by Robert Bierman}

\subsection{Contributions by Victor Woeltjen}

\subsection{Contributions by Kay Choi}

\subsection{Contributions by Steven Gimeno}

\subsection{Contributions by Jason Lum}

\subsection{Contributions by Ying Kit Ng}

\subsection{Contributions by Bianca Uy}


\newpage 
\section{Platform}

\subsection{OpenMRS}

\newpage 
\section{User guide}

\subsection{Rules}

\subsubsection{Creating a rule}

\subsubsection{Modifying an existing rule}

\subsection{Alerts}

\subsubsection{Patient dashboard}

\subsubsection{Patient summary}

\newpage 
\section{Use cases}

\subsection{Rule administration}

\subsection{Alerts}

\newpage 
\section{Sequence diagrams}

\newpage 
\section{Design overview}

	The DSS1 rule subsystem is incorporated into OpenMRS in a simple Client-Server fashion. The target implementation will feature client-side web pages which interact with the DSS1 rule subsystem on the server by way of DSSRuleService. Figure 
~\ref{fig:ARCHITECTURE} illustrates this interaction.

\tikzstyle{layer}=[rectangle, 
                   rounded corners,
                   draw=black, 
                   align=center,
                   anchor=north]
\tikzstyle{communicates}=[draw, ->, >=triangle 60]
\tikzstyle{boundary}=[draw, -, dashed]
\begin{figure}\begin{center}
\begin{tikzpicture}
\node (Client) [layer] { 
  \textbf{Client Pages}      \\
  Patient Summary \\
  Patient Dashboard \\
  DSS Rule Administration
};
\node (Boundary) [below=of Client, minimum width=3in] {};
\node [above=of Boundary.east] {\emph{Client}};
\node [below=of Boundary.east] {\emph{Server}};
\node (Service) [layer, below=of Boundary] { 
  \textbf{DSS Rule Service}  \\
  Run rules  \\
  List rules \\
  Add rules  \\
  Edit rules
};
\node (Interpreter) [layer, below=of Service] { 
  \textbf{DSS1 Interpreter}  \\
  Intrinsics  \\
  Evaluations \\
  Flow control
};
\path [communicates] (Client.300     ) -- (Service.65     ) {};
\path [communicates] (Service.115    ) -- (Client.240     ) {};
\path [communicates] (Service.295    ) -- (Interpreter.60 ) {};
\path [communicates] (Interpreter.120) -- (Service.245    ) {};
\path [boundary] (Boundary.180) -- (Boundary.0    ) {};
%\path [boundary]     (Boundary.west  ) -- (Boundary.east  ) {};
\end{tikzpicture}
\caption{Architecture overview}\label{fig:ARCHITECTURE}
\end{center}\end{figure}

\subsection{Rule service}

\subsubsection{Rule storage}

\subsection{Views}

\subsection{Interpreter}

\subsubsection{Flow control}

\subsubsection{Execution context}

\subsubsection{Evaluation of expressions}

\subsubsection{Intrinsic functions}

% \section{External documentation}

\newpage 
\section{Package diagrams}

\newpage 
\section{Class diagrams}

\tikzstyle{class}=[rectangle, 
                   draw=black, 
                   rectangle split, 
                   rectangle split parts=3,
                   align=center,
                   anchor=north]
\tikzstyle{implements}=[draw, ->, >=open triangle 90]
\tikzstyle{aggregates}=[draw, <-, >=open diamond]
\tikzstyle{contains}=[draw, <-, >=diamond]

\begin{figure}
\begin{tikzpicture}[node distance=0.25in]
\node (ASTVisitor) [class] { 
  \emph{ASTVisitor}
  \nodepart{second}  
  \nodepart[align=justify]{third}
  + visitBlockTree(t : AST) \\
  + visitIdTree(t : AST)    \\
  ...
};
\node (InterpreterVisitor) [class, below=of ASTVisitor] { 
  \textbf{InterpreterVisitor}
  \nodepart{second}  
  \nodepart[align=justify]{third}
};

\node (DSSExecutionContext) [class, right=of InterpreterVisitor] { 
  \textbf{DSSExecutionContext}
  \nodepart{second}  
  \nodepart[align=justify]{third}
  + setConstant(name : String,  value : DSSValue) \\
  + setIntrinsic(name : String,  func : DSSFunction)
};
\node (ExecutionContext) [class, above=of DSSExecutionContext] { 
  \textbf{ExecutionContext}
  \nodepart[align=justify]{second}  
  - evaluator : Evaluator
  \nodepart[align=justify]{third}
  + beginScope() \\
  + endScope() \\
  + getEvaluator() : Evaluator \\
  + getFunction(name : String) : DSSFunction \\
  + getReturnValue() : DSSValue \\
  + setReturnValue(v : DSSValue) \\
  + setFunction(name : String, f : DSSFunction)
};
\node (ASTInterpreter) [class, below=of InterpreterVisitor.south east] { 
  \emph{ASTInterpreter}
  \nodepart{second}  
  \nodepart[align=justify]{third}
  + interpret(ast : AST, 
              c : ExecutionContext, 
              v : ASTVisitor) \\ \hspace{10pt} : Object
};
\node (BlockInterpreter) [class, anchor=west, below=of ASTInterpreter] { 
  \textbf{BlockInterpreter}
  \nodepart{second}  
  \nodepart[align=justify]{third}
};
\node (IdInterpreter) [class, right=of ASTInterpreter] { 
  \textbf{IdInterpreter}
  \nodepart{second}  
  \nodepart[align=justify]{third}
};
\node (NamingContext) [class, above=of ExecutionContext] { 
  \emph{NamingContext}
  \nodepart[align=justify]{third}
  +set (name : String, value : DSSValue) \\
  +get (name : String) : DSSValue
};

\path [implements] (ExecutionContext) -- (NamingContext) {};
\path [implements] (InterpreterVisitor) -- (ASTVisitor) {};
\path [implements] (BlockInterpreter) -- (ASTInterpreter) {};
\path [implements] (IdInterpreter) -- (ASTInterpreter) {};
\path [implements] (DSSExecutionContext) -- (ExecutionContext) {};
\path [contains] (InterpreterVisitor) -- (DSSExecutionContext) {};
\path [contains] (InterpreterVisitor) -- (ASTInterpreter) {};
\draw (InterpreterVisitor) -- (ASTInterpreter) 
      node [midway, right] {\small{1..*}};
\draw (InterpreterVisitor) -- (DSSExecutionContext) 
      node [midway, above] {\small{1..1}};
\end{tikzpicture}
\caption{Interpreter visitor}
\end{figure}

\begin{figure}
\begin{tikzpicture}[node distance=0.33in]
\node (ExecutionContext) [class] { 
  \textbf{ExecutionContext}
  \nodepart[align=justify]{second}  
  - functions : Map\textless String, DSSFunction\textgreater \\
  - variables : Map\textless String, DSSValue\textgreater \\
  - evaluator : Evaluator
  \nodepart[align=justify]{third}
  + beginScope() \\
  + endScope() \\
  + getEvaluator() : Evaluator \\
  + getFunction(name : String) : DSSFunction \\
  + getReturnValue() : DSSValue \\
  + setReturnValue(v : DSSValue) \\
  + setFunction(name : String, f : DSSFunction)
};
\node (DSSValue) [class, right=of ExecutionContext] { 
  \emph{DSSValue}
  \nodepart[align=justify]{second}  
  - timestamp : Date
  \nodepart[align=justify]{third}
  + add (v : DSSValue) : DSSValue \\
  + sub (v : DSSValue) : DSSValue \\
  + div (v : DSSValue) : DSSValue \\
  ... \\
  + getTimeStamp() : Date
};
\node (NamingContext) [class, above=of ExecutionContext] { 
  \emph{NamingContext}
  \nodepart[align=justify]{third}
  +set (name : String, value : DSSValue) \\
  +get (name : String) : DSSValue
};
\node (DSSFunction) [class, right=of NamingContext] { 
  \emph{DSSFunction}
  \nodepart{second}  
  \nodepart[align=justify]{third}
  + call(args : DSSValue[]) : DSSValue \\
  + passAsIdentifier(argIndex : int) \\ \hspace{10pt} : boolean
};

\node (Evaluator) [class, below=of ExecutionContext] { 
  \emph{Evaluator}
  \nodepart{second}  
  \nodepart[align=justify]{third}
  + castTo(javaClass : Class, v : DSSValue) : Object \\
  + evaluate(leftOper : DSSValue, \\ 
    \hspace{10pt} operator : String, \\
    \hspace{10pt} rightOper : DSSValue) : DSSValue\\
  + evaluateLiteral(lit : Symbol) : DSSValue \\
  + newAllocation(fields : String[]) : DSSValue \\
  + toDSSValue(javaObject : Object) : DSSValue
};
\node (DSSEvaluator) [class, below=of Evaluator] { 
  \textbf{DSSEvaluator}
  \nodepart{second}  
  \nodepart[align=justify]{third}
};


\path [aggregates] (ExecutionContext) -- (DSSValue) {};
\path [aggregates] (ExecutionContext) -- (DSSFunction) {};
\path [contains] (ExecutionContext) -- (Evaluator) {};
\path [implements] (ExecutionContext) -- (NamingContext) {};
\path [implements] (DSSEvaluator) -- (Evaluator) {};
\draw (ExecutionContext) -- (DSSValue) 
      node [midway, above] {\small{1..*}};
\draw (ExecutionContext) -- (DSSFunction) 
      node [midway, below] {\small{1..*}};

\end{tikzpicture}
\caption{Execution context}
\end{figure}

\begin{figure}
\begin{tikzpicture}[node distance=0.25in]
\node (DSSValue) [class] { 
  \emph{DSSValue}
  \nodepart[align=justify]{second}  
  - timestamp : Date
  \nodepart[align=justify]{third}
  + add (v : DSSValue) : DSSValue \\
  + sub (v : DSSValue) : DSSValue \\
  + div (v : DSSValue) : DSSValue \\
  + mul (v : DSSValue) : DSSValue \\
  + power (v : DSSValue) : DSSValue \\
  + concat (v : DSSValue) : DSSValue \\
  + and (v : DSSValue) : DSSValue \\
  + or (v : DSSValue) : DSSValue \\
  + equal (v : DSSValue) : boolean \\
  + notequal (v : DSSValue) : boolean \\
  + lessthan (v : DSSValue) : boolean \\
  + greaterthan (v : DSSValue) : boolean \\
  + lessthanequal (v : DSSValue) : boolean \\
  + greaterthanequal (v : DSSValue) : boolean \\
  + getTimeStamp() : Date \\
  + setTimeStamp(date : Date)
};

\node (DSSValueNumeric) [class, right=of DSSValue.north east] {
  \emph{DSSValueNumeric}
  \nodepart[align=justify]{second}  
  \nodepart[align=justify]{third}
};
\node (DSSValueFloat) [class, above=of DSSValueNumeric] {
  \textbf{DSSValueFloat}
  \nodepart[align=justify]{second}  
  - value : double
  \nodepart[align=justify]{third}
};
\node (DSSValueInt) [class, below=of DSSValueNumeric] {
  \textbf{DSSValueInt}
  \nodepart[align=justify]{second}  
  - value : long
  \nodepart[align=justify]{third}
};
\node (DSSValueBool) [class, below=of DSSValueInt] {
  \textbf{DSSValueBool}
  \nodepart[align=justify]{second}  
  - value : boolean
  \nodepart[align=justify]{third}
};
\node (DSSValueString) [class, below=of DSSValueBool] {
  \textbf{DSSValueString}
  \nodepart[align=justify]{second}  
  - value : String
  \nodepart[align=justify]{third}
};
\node (DSSValueList) [class, below=of DSSValueString] {
  \textbf{DSSValueList}
  \nodepart[align=justify]{second}  
  - value : List
  \nodepart[align=justify]{third}
};
\node (DSSValueObject) [class, below=of DSSValue] {
  \textbf{DSSValueObject}
  \nodepart[align=justify]{second}  
  - fields : Map\textless String, DSSValue\textgreater 
  \nodepart[align=justify]{third}
};
\node (DSSValueNull) [class, right=of DSSValueObject] {
  \textbf{DSSValueNull}
  \nodepart[align=justify]{second}  
  \nodepart[align=justify]{third}
};

\node (NamingContext) [class, below=of DSSValueObject] { 
  \emph{NamingContext}
  \nodepart[align=justify]{third}
  +set (name : String, value : DSSValue) \\
  +get (name : String) : DSSValue
};

\path [implements] (DSSValueInt) -- (DSSValueNumeric) {};
\path [implements] (DSSValueFloat) -- (DSSValueNumeric) {};
\path [implements] (DSSValueNumeric) -- (DSSValue) {};
\path [implements] (DSSValueBool) -- (DSSValue) {};
\path [implements] (DSSValueString) -- (DSSValue) {};
\path [implements] (DSSValueList) -- (DSSValue) {};
\path [implements] (DSSValueNull) -- (DSSValue) {};
\path [implements] (DSSValueObject) -- (DSSValue) {};
\path [implements] (DSSValueObject) -- (NamingContext) {};
\end{tikzpicture}
\caption{Values}
\end{figure}

\newpage 
\section{API documentation}

\end{document}